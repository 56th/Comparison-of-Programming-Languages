\newgeometry{left=2cm,right=1cm,top=1cm,bottom=1cm,includefoot,heightrounded}

\section{Указания к выполнению работы}

Разработайте программу для решения поставленной задачи на языке \textbf{С++}, удовлетворяющую следующим требованиям:

\begin{enumerate}
	\item программа должна использовать для ввода-вывода потоки \textbf{STL},
	\item программа не должна содержать собственных реализаций стандартных алгоритмов и структур данных, а использовать существующие в \textbf{STL},
	\item размер каждой подпрограммы не должен превышать 10 строк.
\end{enumerate}

Протестируйте разработанную программу. Исследуйте асимптотические свойства разработанной программы на системе тестов с возрастающей размерностью.

(\url{http://dispace.edu.nstu.ru/didesk/course/show/5626/5})

\subsection{Вариант задания}

Задан большой текст (книга). Для наиболее часто встречающегося слова найти $N$ наиболее часто встречающихся пар слов.

\section{Размышления}

Для поиска наиболее популярного слова (или пары слов) удобно использовать контейнер \href{http://en.cppreference.com/w/cpp/container/map}{std::map} --- ассоциативный массив с красно-чёрным деревом под капюшоном. Ключ суть слово, значение --- частота.

Через $n$ обозначим количество слов в нашей входной книжке. Тогда для занесения в контейнер данных потребуется $O(n \log n)$ времени (поиск в дереве осуществляется за логарифм). 

Для проверки предполагаемой сложности будем урезать длину книжки в 4 раза. Тогда отношение времени работы программы

\begin{equation}
\label{complexity}
	frac{4n \log_2 4n}{n \log_2 n} = 4 ( 2 + frac{\log_2 4n}{\log_2 n})
\end{equation}